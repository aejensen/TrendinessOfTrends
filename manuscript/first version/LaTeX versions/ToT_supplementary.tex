\documentclass[11pt,]{article}
\usepackage{lmodern}
\usepackage{amssymb,amsmath}
\usepackage{ifxetex,ifluatex}
\usepackage{fixltx2e} % provides \textsubscript
\ifnum 0\ifxetex 1\fi\ifluatex 1\fi=0 % if pdftex
  \usepackage[T1]{fontenc}
  \usepackage[utf8]{inputenc}
\else % if luatex or xelatex
  \ifxetex
    \usepackage{mathspec}
  \else
    \usepackage{fontspec}
  \fi
  \defaultfontfeatures{Ligatures=TeX,Scale=MatchLowercase}
\fi
% use upquote if available, for straight quotes in verbatim environments
\IfFileExists{upquote.sty}{\usepackage{upquote}}{}
% use microtype if available
\IfFileExists{microtype.sty}{%
\usepackage{microtype}
\UseMicrotypeSet[protrusion]{basicmath} % disable protrusion for tt fonts
}{}
\usepackage[margin=1in]{geometry}
\usepackage{hyperref}
\hypersetup{unicode=true,
            pdfborder={0 0 0},
            breaklinks=true}
\urlstyle{same}  % don't use monospace font for urls
\usepackage{graphicx,grffile}
\makeatletter
\def\maxwidth{\ifdim\Gin@nat@width>\linewidth\linewidth\else\Gin@nat@width\fi}
\def\maxheight{\ifdim\Gin@nat@height>\textheight\textheight\else\Gin@nat@height\fi}
\makeatother
% Scale images if necessary, so that they will not overflow the page
% margins by default, and it is still possible to overwrite the defaults
% using explicit options in \includegraphics[width, height, ...]{}
\setkeys{Gin}{width=\maxwidth,height=\maxheight,keepaspectratio}
\IfFileExists{parskip.sty}{%
\usepackage{parskip}
}{% else
\setlength{\parindent}{0pt}
\setlength{\parskip}{6pt plus 2pt minus 1pt}
}
\setlength{\emergencystretch}{3em}  % prevent overfull lines
\providecommand{\tightlist}{%
  \setlength{\itemsep}{0pt}\setlength{\parskip}{0pt}}
\setcounter{secnumdepth}{5}
% Redefines (sub)paragraphs to behave more like sections
\ifx\paragraph\undefined\else
\let\oldparagraph\paragraph
\renewcommand{\paragraph}[1]{\oldparagraph{#1}\mbox{}}
\fi
\ifx\subparagraph\undefined\else
\let\oldsubparagraph\subparagraph
\renewcommand{\subparagraph}[1]{\oldsubparagraph{#1}\mbox{}}
\fi

%%% Use protect on footnotes to avoid problems with footnotes in titles
\let\rmarkdownfootnote\footnote%
\def\footnote{\protect\rmarkdownfootnote}

%%% Change title format to be more compact
\usepackage{titling}

% Create subtitle command for use in maketitle
\newcommand{\subtitle}[1]{
  \posttitle{
    \begin{center}\large#1\end{center}
    }
}

\setlength{\droptitle}{-2em}
  \title{Supplementary Material for\\
Quantifying the Trendiness of Trends}
  \pretitle{\vspace{\droptitle}\centering\huge}
  \posttitle{\par}
  \author{Andreas Kryger Jensen and Claus Thorn Ekstrøm\\
Biostatistics, Institute of Public Health, University of Copenhagen\\
\href{mailto:aeje@sund.ku.dk}{\nolinkurl{aeje@sund.ku.dk}},
\href{mailto:ekstrom@sund.ku.dk}{\nolinkurl{ekstrom@sund.ku.dk}}}
  \preauthor{\centering\large\emph}
  \postauthor{\par}
  \predate{\centering\large\emph}
  \postdate{\par}
  \date{26 December, 2019}

\usepackage{bm}
\usepackage{amssymb}
\usepackage[labelfont=bf]{caption}
\DeclareMathOperator*{\argsup}{arg\,sup}
\DeclareMathOperator*{\argmin}{arg\,min}
\DeclareMathOperator*{\E}{E}
\DeclareMathOperator*{\Cov}{Cov}
\DeclareMathOperator*{\Cor}{Cor}
\DeclareMathOperator*{\Var}{Var}
\DeclareMathOperator*{\Erf}{Erf}
\DeclareMathOperator*{\Erfc}{Erfc}
\usepackage{multirow}
\usepackage[amsthm,thmmarks]{ntheorem}
\newtheorem{definition}{Definition}
\newtheorem{assumption}{Assumption}
\theoremsymbol{\ensuremath{\blacksquare}}
\newtheorem{proposition}{Proposition}
\theoremstyle{nonumberplain}
\newtheorem{Proof}{Proof}

\begin{document}
\maketitle

\appendix

\section{Proof of Proposition 1}\label{sec:appendix1}

Let \(\mathbf{Y} = (Y_1, \ldots, Y_n)\) and
\(\mathbf{t} = (t_1, \ldots, t_n)\) be the vectors of observed outcomes
and associated sampling times. From the data generating model we observe
that the marginal distribution of the vector of observed outcomes
\(\mathbf{Y} \mid \mathbf{t}, \Theta\) is

\begin{align*}
P(\mathbf{Y} \mid \mathbf{t}, \Theta) &= \int P(\mathbf{Y} \mid f(\mathbf{t}), \mathbf{t}, \Theta)dP(f(\mathbf{t}) \mid \mathbf{t}, \Theta)\\
  &= N(\mu_\beta(\mathbf{t}), C_\theta(\mathbf{t}, \mathbf{t}) + \sigma^2 I)
\end{align*}

where
\(\mu_\beta(\mathbf{t}) = (\mu_\beta(t_1), \ldots \mu_\beta(t_n))\),
\(C_\theta(\mathbf{t}, \mathbf{t})\) is the \(n \times n\) covariance
matrix obtained by evaluating \(C_\theta(s,t)\) at
\(\{(s,t) \in \mathbf{t} \times \mathbf{t}\}\) and \(I\) is an
\(n \times n\) identity matrix. This implies that the joint distribution
of \(\mathbf{Y}\) and the latent functions \((f, df, d^2\!f)\) evaluated
at an arbitrary vector of time points \(\mathbf{t}^\ast\) is

\begin{align*}
  \begin{bmatrix}f(\mathbf{t}^\ast)\\ df(\mathbf{t}^\ast)\\ d^2\!f(\mathbf{t}^\ast)\\ \mathbf{Y}\end{bmatrix} \mid \mathbf{t}, \Theta \sim N\left(\begin{bmatrix}\mu_\beta(\mathbf{t}^\ast)\\ d\mu_\beta(\mathbf{t}^\ast)\\ d^2\mu_\beta(\mathbf{t}^\ast)\\ \mu_\beta(\mathbf{t})\end{bmatrix}, \begin{bmatrix}C_\theta(\mathbf{t}^\ast,\mathbf{t}^\ast) & \partial_2 C_\theta(\mathbf{t}^\ast, \mathbf{t}^\ast) & \partial_2^2 C_\theta(\mathbf{t}^\ast, \mathbf{t}^\ast) & C_\theta(\mathbf{t}^\ast, \mathbf{t})\\ \partial_1 C_\theta(\mathbf{t}^\ast, \mathbf{t}^\ast) &  \partial_1 \partial_2 C_\theta(\mathbf{t}^\ast, \mathbf{t}^\ast) & \partial_1 \partial_2^2 C_\theta(\mathbf{t}^\ast, \mathbf{t}^\ast) & \partial_1 C_\theta(\mathbf{t}^\ast, \mathbf{t})\\ \partial_1^2 C_\theta(\mathbf{t}^\ast, \mathbf{t}^\ast) & \partial_1^2\partial_2 C_\theta(\mathbf{t}^\ast, \mathbf{t}^\ast) & \partial_1^2\partial_2^2 C_\theta(\mathbf{t}^\ast, \mathbf{t}^\ast) & \partial_1^2 C_\theta(\mathbf{t}^\ast, \mathbf{t})\\ C_\theta(\mathbf{t}, \mathbf{t}^\ast) & \partial_2 C_\theta(\mathbf{t}, \mathbf{t}^\ast) & \partial_2^2 C_\theta(\mathbf{t}, \mathbf{t}^\ast)  & C_\theta(\mathbf{t}, \mathbf{t}) + \sigma^2 I\end{bmatrix}\right)
\end{align*}

where \(\partial_j^k\) denotes the \(k\)'th order partial derivative
with respect to the \(j\)'th variable.

By the standard formula for deriving conditional distributions in a
multivariate normal model, the posterior distribution of
\((f, df, d^2\!f)\) evaluated at the \(p\) time points in
\(\mathbf{t}^\ast\) is

\begin{align*}
\begin{bmatrix}f(\mathbf{t}^\ast)\\ df(\mathbf{t}^\ast)\\ d^2\!f(\mathbf{t}^\ast)\end{bmatrix} \mid \mathbf{Y}, \mathbf{t}, \Theta \sim N\left(\bm{\mu}, \bm{\Sigma}\right)
\end{align*}

where \(\bm{\mu} \in \mathbb{R}^{3p}\) is the column vector of posterior
expectations and \(\bm{\Sigma} \in \mathbb{R}^{3p \times 3p}\) is the
joint posterior covariance matrix, and these are given by

\begin{align*}
  \bm{\mu} &= \begin{bmatrix}\mu_\beta(\mathbf{t}^\ast)\\ d\mu_\beta(\mathbf{t}^\ast)\\ d^2\mu_\beta(\mathbf{t}^\ast)\end{bmatrix} + \begin{bmatrix}C_\theta(\mathbf{t}^\ast, \mathbf{t})\\ \partial_1 C_\theta(\mathbf{t}^\ast, \mathbf{t})\\ \partial_1^2 C_\theta(\mathbf{t}^\ast, \mathbf{t})\end{bmatrix}K_{\theta,\sigma}(\mathbf{t}, \mathbf{t})^{-1} (\mathbf{Y} - \mu_\beta(\mathbf{t}))\\
  \bm{\Sigma} &= \begin{bmatrix}C_\theta(\mathbf{t}^\ast, \mathbf{t}^\ast) & \partial_2 C_\theta(\mathbf{t}^\ast, \mathbf{t}^\ast) & \partial_2^2 C_\theta(\mathbf{t}^\ast, \mathbf{t}^\ast)\\ \partial_1 C_\theta(\mathbf{t}^\ast, \mathbf{t}^\ast) & \partial_1 \partial_2 C_\theta(\mathbf{t}^\ast,\mathbf{t}^\ast) & \partial_1 \partial_2^2 C_\theta(\mathbf{t}^\ast, \mathbf{t}^\ast)\\ \partial_1^2 C_\theta(\mathbf{t}^\ast, \mathbf{t}^\ast) & \partial_1^2 \partial_2 C_\theta(\mathbf{t}^\ast, \mathbf{t}^\ast) & \partial_1^2 \partial_2^2 C_\theta(\mathbf{t}^\ast,\mathbf{t}^\ast)\end{bmatrix} - \begin{bmatrix}C_\theta(\mathbf{t}^\ast, \mathbf{t})\\\partial_1 C_\theta(\mathbf{t}^\ast, \mathbf{t})\\ \partial_1^2 C_\theta(\mathbf{t}^\ast, \mathbf{t})\end{bmatrix}K_{\theta,\sigma}(\mathbf{t}, \mathbf{t})^{-1}\begin{bmatrix}C_\theta(\mathbf{t}, \mathbf{t}^\ast)\\\partial_2 C_\theta(\mathbf{t}, \mathbf{t}^\ast)\\ \partial_2^2 C_\theta(\mathbf{t}, \mathbf{t}^\ast)\end{bmatrix}^T
\end{align*}

where
\(K_{\theta,\sigma}(\mathbf{t}, \mathbf{t}) = C_\theta(\mathbf{t}, \mathbf{t}) + \sigma^2 I\).
Partitioning \(\bm{\mu}\) and \(\bm{\Sigma}\) as

\begin{align*}
  \bm{\mu} = \begin{bmatrix}\mu_f(\mathbf{t}^\ast \mid \Theta)\\ \mu_{df}(\mathbf{t}^\ast \mid \Theta)\\ \mu_{d^2\!f}(\mathbf{t}^\ast \mid \Theta)\end{bmatrix}, \quad \bm{\Sigma} = \begin{bmatrix}\Sigma_{f}(\mathbf{t}^\ast,\mathbf{t}^\ast \mid \Theta) & \Sigma_{f,df}(\mathbf{t}^\ast,\mathbf{t}^\ast \mid \Theta) & \Sigma_{f,d^2\!f}(\mathbf{t}^\ast,\mathbf{t}^\ast \mid \Theta)\\ \Sigma_{f,df}(\mathbf{t}^\ast,\mathbf{t}^\ast \mid \Theta) & \Sigma_{df}(\mathbf{t}^\ast,\mathbf{t}^\ast \mid \Theta) & \Sigma_{df,d^2\!f}(\mathbf{t}^\ast,\mathbf{t}^\ast \mid \Theta)\\ \Sigma_{d^2\!f,f}(\mathbf{t}^\ast,\mathbf{t}^\ast \mid \Theta) & \Sigma_{d^2\!f,df}(\mathbf{t}^\ast,\mathbf{t}^\ast \mid \Theta) & \Sigma_{d^2\!f}(\mathbf{t}^\ast,\mathbf{t}^\ast \mid \Theta)\end{bmatrix}
\end{align*}

and completing the matrix algebra, we obtain the expressions of the
individual components given in the Proposition.

\section{Proof of Proposition 3}\label{sec:appendix2}

Rice showed in section 3.3. of Rice (1945) that the expected number of
zero-crossings of a Gaussian process \(X\) on an interval
\(\mathcal{I}\) is given by

\begin{align}
\int_{\mathcal{I}} \int_{-\infty}^\infty |v|f_{X(t), dX(t)}(0, v)\mathrm{d}v\mathrm{d}t\label{eq:rice}
\end{align}

where \(f_{X(t), dX(t)}\) is the joint density function of \(X\) and its
derivative \(dX\) at time \(t\). To derive the expression for the
Expected Trend Instability we must apply the Rice formula to the joint
posterior distribution of \((df, d^2\!f)\). From Proposition 1 the
distribution of \((df, d^2\!f) \mid \mathbf{Y}, \mathbf{t}, \Theta\) is
bivariate normal for each \(t\).

Let \(\mu_{df}\), \(\mu_{d^2\!f}\), \(\Sigma_{df}\) and
\(\Sigma_{d^2\!f}\) be defined as in Proposition 1 and define further

\begin{align*}
  \omega(t \mid \Theta) = \frac{\Sigma_{df, d^2\!f}(t,t \mid \Theta)}{\Sigma_{df}(t,t \mid \Theta)^{1/2}\Sigma_{d^2\!f}(t,t \mid \Theta)^{1/2}}
\end{align*}

as the posterior point-wise cross-correlation function between \(df\)
and \(d^2\!f\). The joint posterior density function of \((df, d^2\!f)\)
at any time \(t\) evaluated at \((0, v)\) can be factorized as

\begin{align*}
f_{df(t), d^2\!f(t)}(0, v) = c_1(t) e^{c_2(t)} e^{-c_3(t) v^2 - 2c_4(t) v}
\end{align*}

where \(c_1, \ldots, c_4\) are functions of time given by

\begin{align*}
 c_1(t) &= (2\pi)^{-1} \Sigma_{df}(t,t \mid \Theta)^{-1/2}\Sigma_{d^2\!f}(t,t \mid \Theta)^{-1/2} (1-\omega(t \mid \Theta)^2)^{-1/2}\\
 c_2(t) &= \frac{\mu_{df}(t \mid \Theta)^2}{2\Sigma_{df}(t,t \mid \Theta)(\omega(t \mid \Theta)^2 - 1)} + \frac{\mu_{d^2\!f}(t \mid \Theta)^2}{2\Sigma_{d^2\!f}(t,t \mid \Theta)(\omega(t \mid \Theta)^2 - 1)}\\
        &- \frac{\mu_{df}(t \mid \Theta)\mu_{d^2\!f}(t \mid \Theta)\omega(t \mid \Theta)}{\Sigma_{df}(t,t \mid \Theta)^{1/2}\Sigma_{d^2\!f}(t,t \mid \Theta)^{1/2}(\omega(t \mid \Theta)^2 - 1)}\\
 c_3(t) &= -\frac{1}{2}\Sigma_{d^2\!f}(t,t \mid \Theta)^{-1}(\omega(t \mid \Theta)^2-1)^{-1}\\
 c_4(t) &= -\frac{\mu_{df}(t \mid \Theta) \Sigma_{d^2\!f}(t,t \mid \Theta)^{1/2} \omega(t \mid \Theta) - \mu_{d^2\!f}(t \mid \Theta) \Sigma_{df}(t,t \mid \Theta)^{1/2}}{2\Sigma_{d^2\!f}(t,t \mid \Theta)(\omega(t \mid \Theta)^2-1)\Sigma_{df}(t,t \mid \Theta)^{1/2}}
\end{align*}

Let \(d\mathrm{ETI}(t \mid \Theta)\) denote the inner integral in
Equation (\ref{eq:rice}). Using the factorization of the joint posterior
density we may write it was

\begin{align}
\begin{split}
d\mathrm{ETI}(t \mid \Theta) &= \int_{-\infty}^\infty |v| f_{df(t), d^2\!f(t)}(0, v)\mathrm{d}v\\
 &= c_1(t) e^{c_2(t)}\int_{-\infty}^\infty |v| e^{-c_3(t) v^2 - 2c_4(t) v}\mathrm{d}v\\
 &= c_1(t) e^{c_2(t)}\left(\int_0^\infty v e^{-c_3(t) v^2 + 2c_4(t) v}\mathrm{d}v + \int_0^\infty v e^{-c_3(t) v^2 - 2c_4(t) v}\mathrm{d}v\right)
\end{split}
\label{dETIintegral1}
\end{align}

Because \(c_3(t) > 0\) for all t since
\(\Sigma_{d^2\!f}(t,t \mid \Theta) > 0\) and
\(|\omega(t \mid \Theta)| < 1\) by Assumption A4 we obtain the following
solution for the type of integral in the previous display by using
formula 5 in section 3.462 on page 365 of Gradshteyn and Ryzhik (2014)

\begin{align}
\int_{0}^\infty v e^{-c_3(t) v^2 \pm 2c_4(t) v}\mathrm{d}v = \frac{1}{2c_3(t)} \pm \frac{c_4(t)}{2c_3(t)}\frac{\pi^{1/2}}{c_3(t)^{1/2}}e^{\frac{c_4(t)^2}{c_3(t)}}\left(1 \pm \Erf\left(\frac{c_4(t)}{\sqrt{c_3(t)}}\right)\right)\label{dETIintegral2}
\end{align}

where \(\Erf\colon\, x \mapsto 2\pi^{-1}\int_0^x e^{-u^2}\mathrm{d}u\)
is the error function. Combining Equations (\ref{dETIintegral1}) and
(\ref{dETIintegral2}) we may express \(d\mathrm{ETI}\) as

\begin{align*}
d\mathrm{ETI}(t \mid \Theta) &= c_1(t) e^{c_2(t)}\left(\frac{1}{c_3(t)} + \frac{c_4(t)}{c_3(t)}\frac{\pi^{1/2}}{c_3(t)^{1/2}}e^{\frac{c_4(t)^2}{c_3(t)}}\Erf\left(\frac{c_4(t)}{\sqrt{c_3(t)}}\right)\right)
\end{align*}

Defining \(\zeta(t \mid \Theta) = \sqrt{2}c_4(t)c_3(t)^{-1/2}\) and
collecting some terms, the index can be rewritten as

\begin{align*}
d\mathrm{ETI}(t \mid \Theta) &= \frac{c_1(t)}{c_3(t)}\left(e^{c_2(t)} + \frac{\pi^{1/2}}{2^{1/2}} e^{\frac{c_4(t)^2}{c_3(t)} + c_2(t)} \zeta(t) \Erf\left(\frac{\zeta(t \mid \Theta)}{2^{1/2}}\right)\right)
\end{align*}

Straightforward arithmetic calculations show that

\begin{align*}
  \frac{c_4(t)^2}{c_3(t)} + c_2(t) = -\frac{\mu_{df}(t \mid \Theta)^2}{2\Sigma_{df}(t,t \mid \Theta)}, \quad c_2(t) = - \frac{1}{2}\left(\zeta(t \mid \Theta)^2 + \frac{\mu_{df}(t \mid \Theta)^2}{\Sigma_{df}(t,t \mid \Theta)}\right)
\end{align*}

and by defining \(\phi\colon\, x \mapsto (2\pi)^{-1/2}e^{-x^2}\) as the
density function of the standard normal distribution we may write
\(e^{\frac{c_4(t)^2}{c_3(t)} + c_2(t)} = (2\pi)^{1/2}\phi\left(\frac{\mu_{df}(t \mid \Theta)}{\Sigma_{df}(t,t \mid \Theta)^{1/2}}\right)\)
and
\(e^{c_2(t)} = 2\pi\phi(\zeta(t))\phi\left(\frac{\mu_{df}(t \mid \Theta)}{\Sigma_{df}(t,t \mid \Theta)^{1/2}}\right)\)
which leads to

\begin{align*}
d\mathrm{ETI}(t \mid \Theta) = \frac{c_1(t)}{c_3(t)}\pi\phi\left(\frac{\mu_{df}(t \mid \Theta)}{\Sigma_{df}(t,t \mid \Theta)^{1/2}}\right)\left(2\phi(\zeta(t \mid \Theta)) + \zeta(t \mid \Theta)\Erf\left(\frac{\zeta(t\mid \Theta)}{2^{1/2}}\right)\right)
\end{align*}

Standard arithmetics show that

\begin{align*}
\frac{c_1(t)}{c_3(t)} =  \frac{1}{\pi}\frac{\Sigma_{d^2\!f}(t,t \mid \Theta)^{1/2}}{\Sigma_{df}(t,t \mid \Theta)^{1/2}}\left(1-\omega(t \mid \Theta)^2\right)^{1/2}
\end{align*}

and we finally obtain the expression

\begin{align*}
d\mathrm{ETI}(t \mid \Theta) = \lambda(t \mid \Theta)\phi\left(\frac{\mu_{df}(t \mid \Theta)}{\Sigma_{df}(t,t \mid \Theta)^{1/2}}\right)\left(2\phi(\zeta(t \mid \Theta)) + \zeta(t \mid \Theta)\Erf\left(\frac{\zeta(t \mid \Theta)}{2^{1/2}}\right)\right)
\end{align*}

where \(\lambda\) and \(\zeta\) are given by

\begin{align*}
\lambda(t \mid \Theta) &= \frac{\Sigma_{d^2\!f}(t,t \mid \Theta)^{1/2}}{\Sigma_{df}(t,t \mid \Theta)^{1/2}}\left(1-\omega(t \mid \Theta)^2\right)^{1/2}\\
  \zeta(t \mid \Theta) &= \frac{\mu_{df}(t \mid \Theta)\Sigma_{d^2\!f}(t,t \mid \Theta)^{1/2}\omega(t)\Sigma_{df}(t,t \mid \Theta)^{-1/2} - \mu_{d^2\!f}(t \mid \Theta)}{\Sigma_{d^2\!f}(t,t \mid \Theta)^{1/2}\left(1 - \omega(t \mid \Theta)^2\right)^{1/2}}
\end{align*}

By definition

\begin{align*}
  \mathrm{ETI}(\mathcal{I} \mid \Theta) = \int_{\mathcal{I}} d\mathrm{ETI}(t \mid \Theta)\mathrm{d}t
\end{align*}

which completes the proof.

\section{Zero-crossings of $f$ and $df$ in the zero-mean stationary case}\label{sec:appendix3}

Let \(f \sim \mathcal{GP}\left(0, C_\theta(\cdot, \cdot)\right)\) where
the \(C_\theta\) is either the Squared Exponential or Rational Quadratic
covariance function. We look at the expected number of zero-crossings on
an interval by either \(f\) and \(df\) as given by the Rice formula in
Equation (\ref{eq:rice}) with either \(X(t) = f(t)\) or
\(X(t) = df(t)\). In this case the expressions simplifies immensely due
to the zero means of both \(f\), \(df\), and \(d^2\!f\) and because
\(\Cov[f(t), df(t)] = 0\) and \(\Cov[df(t), d^2\!f(t)] = 0\). The latter
is a result of using a stationary covariance function for the prior
distribution of \(f\) (Cramer and Leadbetter 1967). In this stationary
case local expected number of zero-crossing of \(f\) and \(df\) are
given by

\begin{align*}
  \frac{\partial_1 \partial_2 C_\theta(s,t)\Bigr|_{\substack{s=t}}^{1/2}}{\pi C_\theta(t,t)^{1/2}} \quad \text{and} \quad   \frac{\partial_1^2 \partial_2^2 C_\theta(s,t)\Bigr|_{\substack{s=t}}^{1/2}}{\pi \partial_1 \partial_2 C_\theta(s,t)\Bigr|_{\substack{s=t}}^{1/2}}
\end{align*}

respectively. It then follows that

\begin{alignat*}{3}
 C_\theta^\text{SE}(t,t) &= \sigma^2, &\quad \partial_1\partial_2 C_\theta^\text{SE}(s,t)\Bigr|_{\substack{s=t}} &= \frac{\sigma^2}{\rho^2}, &\quad \partial_1^2\partial_2^2 C_\theta^\text{SE}(s,t)\Bigr|_{\substack{s=t}} &= \frac{3\sigma^2}{\rho^4}\\   
 C_\theta^\text{RQ}(t,t) &= \sigma^2, & \partial_1\partial_2 C_\theta^\text{RQ}(s,t)\Bigr|_{\substack{s=t}} &= \frac{\sigma^2}{\rho^2}, & \partial_1^2\partial_2^2 C_\theta^\text{RQ}(s,t)\Bigr|_{\substack{s=t}} &= \frac{2\sigma^2 (1+\nu)}{\nu \rho^4}
\end{alignat*}

and the local expected number of zero-crossings of \(f\) and \(df\) for
either the Squared Exponential and the Rational Quadratic covariance
functions are

\begin{alignat*}{2}
 \frac{\partial_1 \partial_2 C_\theta^\text{SE}(s,t)\Bigr|_{\substack{s=t}}^{1/2}}{\pi C_\theta^\text{SE}(t,t)^{1/2}} &= \frac{1}{\pi\rho}, &\qquad \frac{\partial_1^2 \partial_2^2 C_\theta^\text{SE}(s,t)\Bigr|_{\substack{s=t}}^{1/2}}{\pi \partial_1 \partial_2 C_\theta^\text{SE}(s,t)\Bigr|_{\substack{s=t}}^{1/2}} &= \frac{3^{1/2}}{\pi\rho}\\  
 \frac{\partial_1 \partial_2 C_\theta^\text{RQ}(s,t)\Bigr|_{\substack{s=t}}^{1/2}}{\pi C_\theta^\text{RQ}(t,t)^{1/2}} &= \frac{1}{\pi\rho}, & \frac{\partial_1^2 \partial_2^2 C_\theta^\text{RQ}(s,t)\Bigr|_{\substack{s=t}}^{1/2}}{\pi \partial_1 \partial_2 C_\theta^\text{RQ}(s,t)\Bigr|_{\substack{s=t}}^{1/2}} &= \frac{3^{1/2}}{\pi\rho}\left(1 + v^{-1}\right)^{1/2}
\end{alignat*}

\section*{Bibliography}\label{bibliography}
\addcontentsline{toc}{section}{Bibliography}

\hypertarget{refs}{}
\hypertarget{ref-cramer1967stationary}{}
Cramer, Harald, and M. R. Leadbetter. 1967. \emph{Stationary and Related
Stochastic Processes -- Sample Function Properties and Their
Applications.} John Wiley \& Sons, Inc.

\hypertarget{ref-gradshteyn2014table}{}
Gradshteyn, Izrail Solomonovich, and Iosif Moiseevich Ryzhik. 2014.
\emph{Table of Integrals, Series, and Products}. Academic Press.

\hypertarget{ref-rice1945mathematical}{}
Rice, Stephen O. 1945. ``Mathematical Analysis of Random Noise, II.''
\emph{Bell System Technical Journal} 24 (1): 46--156.


\end{document}
